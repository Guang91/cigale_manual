\section{Overview}\label{sec:overview}
Code Investigating GALaxy Emission ({\sc cigale}) is a Python code for the fitting of spectral energy distribution (SED) of galaxies. 
It has been developed for more than 1.5 decades \citep[e.g.][]{burgarella05, noll09, serra11, roehlly14, boquien19}. 
The detailed algorithm is described in \cite{boquien19}.
\cite{yang20} upgraded \xcig\ to allow it fitting X-ray data, and this version is dubbed as ``{\sc x-cigale}''.
We further merged {\sc x-cigale} into the main branch of \xcig\ as well as implemented many improvements and functionalities \citep{yang22}.
The new version is marked as v2022.0.
This manual serves as a ``quick and practical'' reference for the user.
Further questions can be asked in our discussion forum (\url{https://github.com/mboquien/cigale/discussions}).
All materials of \xcig\ (including this manual) can be found on \url{https://cigale.lam.fr}.

In \S\ref{sec:install}, we describe the installation procedures.
\xcig\ has two working modes.
One is fitting the observed galaxy SEDs, and the other is simulating model SEDs. 
These two modes are described in \S\ref{sec:pdf} and \S\ref{sec:saveflux}, respectively. 
Appendix~\hyperref[app:par]{A} lists the main model parameters.
Appendix~\hyperref[app:file]{B} describes the supplementary files used in this manual. 
%Appendix~\ref{app:faq} includes the frequently asked questions.


