\section{Installation}\label{sec:install}
The easiest way is to use pip installation.
To do this, in the downloaded \xcig\ directory,
simply run \\
\$ \textit{pip install .} \\

However, the pip installation above only allows you to use the default downloaded code. 
If you want to modify the code to serve your own research interest, you can install from source. 
The instruction is detailed at \url{https://github.com/mboquien/cigale/discussions/2}

%We assume you have Python 3.4 or higher working on your computer.
%We recommend using ANACONDA to install Python~3 (\url{https://www.anaconda.com/distribution/}).

%\subsection{Method I: pip}
%\begin{enumerate}
%    \item Download the experimental wheel (.whl file) from \url{https://cigale.lam.fr/}
%    \item Open terminal and run \\
%        \$ \textit{pip install pcigale-2018.0.1-py3-none-any.whl} (or any other version of .whl file) \\
%         to install \xcig. You can delete the .whl file after installation.
%\end{enumerate}

%\subsection{Method II: from source}
%\begin{enumerate}
%    \item Download the source file from \url{https://gitlab.lam.fr/gyang/cigale/tree/xray} and decompress
%    \item Install dependencies. Open terminal and run:
%    \begin{itemize}
%        \item[\$] \textit{conda install astropy numpy scipy sqlite sqlalchemy matplotlib configobj}
%    \end{itemize}
%    \item \$ \textit{cd pcigale} (or whatever the name or the directory is)
%    \item If you want to add your own filters for the SED analysis or the creation of models in filters, it is the correct time to do so.
%          Just add your filters into the directory:./database\_builder/filters/.
%          A minimum list of filter transmissions is provided with the \xcig\ distribution downloaded.
%    \item \$ \textit{Python setup.py build} \\
%          Note that if you already have the pcigale/data/data.db file, you need to delete it before re-building.
%    \item \$ \textit{Python setup.py develop} \\
%            will install \xcig. 
%            Note that the pcigale directory should not be removed or \xcig\ will no longer works.
%\end{enumerate}